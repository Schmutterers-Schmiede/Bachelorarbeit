\chapter{Introduction}

\begin{german}
In der Modernen Web-Frontendentwicklung verwenden die meisten Entwickler Frameworks, wie zum Beispiel Angular, React, oder Vue. Grundsätzlich verwenden diese Frameworks alle sogenannte Web-Komponenten:  voneinander unabhängige Blöcke von HTML, CSS und Typescript, welche einen eigenen HTML-tag bekommen. Die Idee dahinter ist es, Codeblöcke, die mehrmals in einem Projekt verwendet werden (z.B. Produktkarten in einem Webshop) in wiederverwendbaren Komponenten zusammenzufassen. Das reduziert die Menge an Code in einem Projekt und erleichtert das Schreiben von „dry“ Code.
Da man Allerdings nicht jedes Projekt mit dem selben Framework umsetzen wird, und alle diese Frameworks Komponenten als grundlegende Bausteine verwenden, wäre es nützlich, Framework-agnostische Komopnenten erstellen zu können, welche dann in jedem beliebigen Framework verwendet werden könnten. Dadurch müsste man Web-Komponenten nur einmal erstellen, anstatt für jedes Framework einzeln.
Das Ziel dieser Arbeit ist es daher, festzustellen, ob es möglich ist, Frameworkagnostische Webkomponenten zu erstellen und diese in verschiedenen Frameworks ohne viel Boiler Plate Code zu verwenden.Als Beispiel wird Stenciljs verwendet, um Komponenten zu erstellen und die Frameworks Angular, React und Vue werden dienen Dazu, die Universale Verwendbarkeit der Komponenten zu demonstrieren.
Neben einigen Testseiten, die die in Stenciljs erstellten Componenten beinhalten wird auch ein Projekt Teil dieser Arbeit sein. Es geht dabei um eine Tablet-app für Therapeuten, welche Essenzielle Patientendaten schnell und einfach zugänglich macht und das Patientenmanagement unterstützt.

\end{german}