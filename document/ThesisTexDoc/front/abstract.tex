\chapter{Abstract}
\label{cha:abstract}

In modern frontend web development, most developers use frameworks like Angular, React or Vue. On a basic level, all of these frameworks use so called web components: bundles of HTML, CSS and Typescript which are independent of the rest of the code and each represented by their own custom HTML tag. The idea is to pack chunks of code which are frequently used throughout a project (i.e. Product cards in a webshop) into reusable components. This reduces the amount of code needed, eliminates the necessety for copy-pasting code segements throughout a web page and promotes “dry” code. 
However, since not every project will be built using the same framework and all these frameworks use components as their basic building blocks, it would be usefull to be able to build framework agnostic web components that can then be used by any framework. This way, components would only need to be created once instead of creating them in each framework separately. 
The subject of this Thesis is therefore to answer the question of whether it is possible to create framework agnostic web components and use them in multiple frameworks without the need for considerable amounts of extra code and to shed some light on how this universal compatibility is achieved. As an example, StencilJS will be used to create web components and the frameworks Angular, React and Vue will be used to demonstrate the universal compatibility. 
Apart from small example pages that contain Components created in StencilJs, a part of this Thesis is also going to be a real-world project; A tablet app for Therapists that displays essential patient data and helps with managing patients. 


